\documentclass[12pt]{article}

\usepackage[utf8]{inputenc}
\usepackage[T1]{fontenc}
\usepackage{amsmath}
\usepackage{hyperref}


\hypersetup{colorlinks=true, citecolor=blue}

\begin{document}

\title{Multi-space Variational Encoder-Decoders for Semi-supervised Labeled Sequence Transduction}
\author{}
\date{}
\maketitle

\section{Idea}
  The general idea seems similar to style-transfer in text. Labeled sequence transduction is just a roundabout way of saying that a source text $x^{(s)}$ is to be transformed into a target text $x^{(t)}$ such that $x^{(t)}$ is conditioned on the labels $y^{(t)}$.

\section{Background}
  The morphological re-inflection problem tries to change a sequence of characters of an inflected word. For example, convert `playing' into `played', given a set of labels $y^{(t)}$ such that $y^{(t)}_{pos}=verb$ and $y^{(t)}_{tense}=past$

\section{Method}
  \begin{itemize}
    \item MSVEDs (system proposed by this paper) use multiple discrete and continuous latent variables in a character based recurrent network to model the transduction of inflected words into their re-inflected forms.
    \item The basic architecture emulates a variational autoencoder. The encoder and decoder parameters are learned in the standard way, by maximizing the variational lower bound on the marginal log-likelihood of the data. The neural network parameterizes both the encoder and decoder weights and back-propagation is done using the VAE re-parameterization trick.
    \item No explicit modeling seems to have been done to disentangle the latent representation $z$ from the labels $y$
    \item Both encoder and decoder architectures are RNNs.
    \item The paper also describes a semi-supervised version of the architecture in which the labels $y$ are inferred directly by a discriminative classifier on the target sequence $x^{(t)}$. This discriminative classifier is trained on labeled instances and is parameterized by an MLP that shares the initial layers of weights with the encoder of the network. This MLP culminates in a Gumbel Softmax layer. The Gumbel softmax layer is used to obtain a continuous approximation of the latent variables, and the probability distribution varies with the temperature $\tau$ where $\tau = 0$ implies a one-hot encoded softmax, and $\tau > 0$ implies a less confident softmax output.
    \item The KL-term is gradually annealed from 0 to a $\lambda_m$, to prevent the latent representation collapsing into the Gaussian prior.
    \item Recurrent dropout is used in the form of omitting random words, and forcing the decoder to rely on the latent representation instead of just the ground-truth at each time-step.        
  \end{itemize}

\section{Observations}
  \begin{itemize}
    \item The experimental observations show that the bi-directional encoder/decoder model works better than the single directional model.
    \item The results also indicate a performance boost when the model is trained using unlabeled sequences, but the growth rate of performance gains grows slower as more data is added.
    \item The conventional method performs well when the re-inflection just involves suffixing the original word, but the MSVED offers better generalized performance, when the inflected word needs to be lemmatized and then augmented.
  \end{itemize}

\end{document}
