\documentclass[12pt]{article}

\usepackage[utf8]{inputenc}
\usepackage[T1]{fontenc}
\usepackage{amsmath}
\usepackage{hyperref}


\hypersetup{colorlinks=true, citecolor=blue}

\begin{document}

\title{Fader Networks: Manipulating Images by Sliding Attributes}
\author{}
\date{}
\maketitle

\section{Idea}
  The authors attempt to disentangle facial features from images and re-generate images after tuning (fader knobs) certain continuous-valued attributes of the image like age, expression, gender etc. This is an encoder-decoder architecture.

  The major difference touted compared to existing methods is that adversarial training is used to learn the latent space, as opposed to the decoder output, thus, helping the latent space become invariant to the attributes (conditioning labels).

\section{Method}
  \begin{itemize}
    \item The attributes that are binary during train time, can be treated as continuous during image generation.
    \item The data set is pictures of actors with certain attributes, like `smile', `glasses', `mouth-open' etc.
    \item The architecture comprises of 3 main components, the encoder, the discriminator and the decoder. The discriminator is the adversarial component.
    \item The discriminator is trained with a single objective in mind: to correct identify the attributes, given an encoded image representation
    \item The encoder-decoder is trained with 2 objectives in mind:
    \begin{itemize}
      \item The decoder being able to reconstruct the original input, given the encoded representation and the true attributes.
      \item The encoded representation making it difficult for the discriminator to ascertain which attributes are present in the original image.
    \end{itemize}
    \item Without the adversarial component, the decoder learns to ignore the true attributes, and changing these at test time for conditioned generation makes no difference to the decoder output, which we don't want.
    \item The cost attributed by the discriminator to the encoder loss is gradually increased from 0 over the course of the training.
    \item The encoded image representation is generated by a convolutional network.
    \item Augmentation of the face images is done by flipping the images horizontally.
    \item The generated images were evaluated qualitatively and quantitatively for naturalness.      
  \end{itemize}

\section{Observations}
  \begin{itemize}
    \item The objective is to make the attributes the only source of information for the extra image attributes.
    \item Avoiding having an adversarial network as part of the decoder is that backpropagation can occur even for discrete objectives, like text sequence prediction.
  \end{itemize}

\end{document}
