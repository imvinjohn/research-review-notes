\documentclass[12pt]{scrartcl}

\usepackage[utf8]{inputenc}
\usepackage[T1]{fontenc}
\usepackage{amsmath}
\usepackage{hyperref}


\hypersetup{
    colorlinks=true,
    citecolor=blue
}

\begin{document}

\title{Improved Variational Autoencoders for Text Modeling using Dilated Convolutions}
\author{}
\date{}
\maketitle

\section{Main Idea}
  The paper \cite{yang2017improved} presents an alternative architecture to LSTM based VAEs. As shown in a previous paper, LSTM-VAEs don't have a significant advantage over LSTM language model \cite{bowman2016generating}. The authors address this by using a dilated CNN decoder to vary the conditioning context of the decoder. The hypothesis is the the typical collapse of the loss function in favor of the KL-divergence term could be addressed by varying the contextual capacity of the decoder.

\section{Method}
  \begin{itemize}
    \item The authors use a typical LSTM based encoder model, use a dilated CNN as as the decoder of the VAE, since previous work has shown that LSTM decoders don't have a significant advantage over non-VAE LSTM language models.
    \item The architecture of the encoder doesn't matter as long as the posterior of the latent representation resembles a Gaussian with unit variance.
    \item The idea of dilated CNNs was introduced with the intention of supplying varying contexts of words as features. As opposed to dense convolutions, dilated convolution skip time-steps to increase the receptive field of the operation.
    \item Dilated convolutions increase the receptive field without increasing the computational costs. Dilations effectively introduce holes in a convolutional operation to be able to expand quickly.
    \item It is okay for the posterior (latent representation) to not completely mimic the Gaussian prior. This will ensure that the space of the latent probabilities offer reasonable generative properties.
    \item Residual blocks are used for faster convergence and to enable building deeper architectures.
    \item Predictions at each step of the decoder is conditioned on the convolutional features concatenated with the latent variable $z$.
    \item The Gumbel softmax function is used as a continuous approximation of an otherwise discrete latent variable, in the framework for semi-supervised text classification.
    \item For unsupervised clustering, the authors still use a discrete label $y$ to encode some information about an unlabeled text $x$, and the discrete label is then used to cluster the $x$ datapoints.
    \item The authors use an LSTM encoder to obtain the latent representation $z$, followed by the dilated CNN to decode. The LSTM encoder is shared by the classifier (discriminator), since the hidden state is fed to an MLP architecture to obtain a classification.
  \end{itemize}

\section{Observations}
  \begin{itemize}
    \item 
  \end{itemize}

\bibliographystyle{unsrt}
\bibliography{improved-variational-autoencoders-for-text-modeling-using-dilated-convolutions}

\end{document}
